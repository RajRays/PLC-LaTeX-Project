\documentclass{beamer}

\mode<presentation> {

\usetheme{Madrid}

\usecolortheme{whale}

}

\usepackage{booktabs}

%----------------------------------------------------------------------------------------
%	TITLE PAGE
%----------------------------------------------------------------------------------------

\title[LaTeX]{An Introduction to LaTeX}

\author[Programming Language Concepts - CSCI 318]{Rajendra Bhagroo, Chris Guevarra, Kim Myeongkeun, Thomas Lavenziano}

\institute[NYIT]
{
New York Institute of Technology \\
\medskip
}
\date{\today}

\begin{document}

\begin{frame}
\titlepage
\end{frame}

%----------------------------------------------------------------------------------------
%	What is LaTeX? (Slide 1)
%----------------------------------------------------------------------------------------

\section{What is LaTeX?}

\subsection{Overview of LaTeX}

\begin{frame}
\frametitle{What is LaTeX?}

LaTeX is a document preparation typesetting system that uses markup tags to create structure within a document.

\\~\\

LaTeX internally uses TeX is its typesetting engine and supports use of various packages. 

\\~\\

A Package named "BEAMER" was used to create this presentation.

\end{frame}

%----------------------------------------------------------------------------------------
%	Features of LaTeX (Slide 2)
%----------------------------------------------------------------------------------------

\subsection{Features of LaTeX}

\begin{frame}
\frametitle{Features of LaTeX}

LaTeX is primarily known for its rich set of features that allow you to format text in specials ways to create objects such as ...

\\~\\

\begin{itemize}
\item Blocks
\item Tables
\item Formulas
\item Theorems
\item Bullets
\item Numbered Bullets
\item SLIDE 9
\item SLIDE 10
\end{itemize}
\end{frame}

%----------------------------------------------------------------------------------------
%	Blocks (Slide 3)
%----------------------------------------------------------------------------------------

\begin{frame}
\frametitle{Blocks}

\begin{block} {Block Syntax}

\textbackslash begin \{block\} \{Block Name\}
\\~\\
    Text Within Block
\\~\\
\textbackslash end \{block\}

\end{block}    

\\~\\


\begin{block}{Block Usage}
Blocks can be used to place emphases on a certain formula or point
\end{block}

\begin{block}{Pythagorean Theorem}
\[ A^2 + B^2 = C^2 \]
\end{block}

\end{frame}

%----------------------------------------------------------------------------------------
%	Tables (Slide 4)
%----------------------------------------------------------------------------------------

\begin{frame}
\frametitle{Tables}

Here is an example of a table in LaTeX

\\~\\

\begin{table}
\begin{tabular}{l l l 1}
\toprule
\textbf{Day1} & \textbf{Day2} & \textbf{Day3} & \textbf{Day4} \\
\midrule
100°F & 91°F & 88°F & 86°F \\
 87°F & 83°F & 84°F & 83°F \\
 80°F & 77°F & 75°F & 79°F \\
\bottomrule
\end{tabular}
\caption{Temperatures Throughout The Day}
\end{table}
\end{frame}

%----------------------------------------------------------------------------------------
%	Formulas (Slide 5)
%----------------------------------------------------------------------------------------

\begin{frame}
\frametitle{Formulas}

LaTeX allows you to write simple, yet elegant formulas

\\~\\

\begin{theorem}[Albert Einstein’s Famous Formula]
$E = mc^2$
\end{theorem}
\end{frame}

%----------------------------------------------------------------------------------------
%	Theorems (Slide 6)
%----------------------------------------------------------------------------------------

\begin{frame}[fragile]
\frametitle{How To Add Theorems}

Theorems allow for longer formulas and text blocks, even allowing native TeX code to be represented!

\\~\\

This verbatim box allows us to see the code used to create the previous Theorem slide

\\~\\

\begin{example}[Theorem Slide Code]
\begin{verbatim}
	\begin{frame}
	\frametitle{Theorem}
	\begin{theorem}[Albert Einstein’s Famous Formula]
	$E = mc^2$
	\end{theorem}
	\end{frame}\end{verbatim}
\end{example}
\end{frame}

%----------------------------------------------------------------------------------------
%	Bullets (Slide 7)
%----------------------------------------------------------------------------------------

\begin{frame}
\frametitle{Bullets}

Bullets
\end{frame}

%----------------------------------------------------------------------------------------
%	Numbered Bullets (Slide 8)
%----------------------------------------------------------------------------------------

\begin{frame}
\frametitle{Numbered Bullets}

Numbered Bullets
\end{frame}

%----------------------------------------------------------------------------------------
%	? (Slide 9)
%----------------------------------------------------------------------------------------

\begin{frame}
\frametitle{Slide 9}

Slide 9
\end{frame}

%----------------------------------------------------------------------------------------
%	? (Slide 10)
%----------------------------------------------------------------------------------------

\begin{frame}
\frametitle{Slide 10}

Slide 10
\end{frame}


%----------------------------------------------------------------------------------------
%	END
%----------------------------------------------------------------------------------------

\end{document} 